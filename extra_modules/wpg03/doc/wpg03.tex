\newcommand{\cpos}{\V{c}}
\newcommand{\cvel}{\dotV{c}}
\newcommand{\cacc}{\ddotV{c}}
\newcommand{\cjerk}{\dddotV{c}}

\newcommand{\cPos}{\V{C}}
\newcommand{\cVel}{\dotV{C}}
\newcommand{\cAcc}{\ddotV{C}}
\newcommand{\cJerk}{\dddotV{C}}

\newcommand{\cstate}{\hatV{c}}
\newcommand{\cState}{\hatV{C}}
%\newcommand{\cstate}{\V{\varsigma}}
%\newcommand{\cState}{\V{\zeta}}

\newcommand{\cop}[1][{}]{{\V[#1]{z}}}
\newcommand{\dcop}[1][{}]{{\dotV[#1]{z}}}
%\newcommand{\cop}{\V{z}}
\newcommand{\CoP}{\V{Z}}
\newcommand{\dCoP}{\dotV{Z}}

\newcommand{\fph}{\hatV{p}}
\newcommand{\FPh}{\hatV{P}}

\newcommand{\fp}[1][{}]{{\V[#1]{p}}}
\newcommand{\FP}{\V{P}}

\newcommand{\fd}{\V{f}}
\newcommand{\FD}{\V{F}}

\newcommand{\cp}{\V{\xi}}


\newcommand{\fCoplanar}{\mathbf{f}_i}
\newcommand{\fExt}{\mathbf{f}_{ext}}
\newcommand{\tExt}{\mathbf{n}_{ext}}
\newcommand{\gravity}{\mathbf{g}}
\newcommand{\comP}{\mathbf{c}}
\newcommand{\comA}{\mathbf{\ddot{c}}}
\newcommand{\momAngD}{\mathbf{\dot{L}}}
\newcommand{\pCoplanar}{\mathbf{p}_i}
\newcommand{\zmp}{\mathbf{z}}


%%%
\chapter{WPG v.03: WPG considering external forces}

\gbox{
    \red{Warning:}
    This module is unsupported -- the documentation may contain errors. Refer
    to \cite{Agravante2016preprint} and \cite{Agravante2016icra} for more
    information.
}

This version of the pattern generator is based on a model that considers external forces.
%%%%%%%%%%%%%%%%%%%%%%%%%%%%%%%%%%%%%%%%%%%%%%%%%%%%%%%%%%%%%%%%%%%%%%%%%%%%%%%%%%%%%%%%%%%%%%%%%%%%%%
\section{Model}
Separating the coplanar contact forces $\fCoplanar$ and an external wrench in the CoM frame:
$[\fExt^\top ~ \tExt^\top]^\top$, the equations of motion are:

\begin{align}
 m(\comA + \gravity) &= \fExt + \sum_i \fCoplanar  \\
 \momAngD &= \tExt + \sum_i (\pCoplanar - \comP) \times \fCoplanar
\end{align}

By following the derivations:
\begin{equation}
 m \comP \times (\comA + \gravity) + \momAngD - \tExt - (\comP \times \fExt)
 =
 \sum_i \pCoplanar \times \fCoplanar
\end{equation}

Dividing to get the Center of Pressure expression on the right hand part:
\begin{equation}
 \frac{m \comP \times (\comA + \gravity) + \momAngD - \tExt - (\comP \times \fExt) }
 {m(\ddot{c}^z + g^z) - f_{ext}^z}
 =
 \frac{\sum_i \pCoplanar \times \fCoplanar}
 {\sum_i f_i^z}
\end{equation}

For a flat ground $p_i^z = 0$ and $\gravity^{x,y} = \mathbf{0}$ and using the approximations: $\ddot{c}^z = 0$ and $\momAngD = 0$ the ZMP in the
plane is:
\begin{equation}
 \zmp^{x,y}
 =
 \frac{\sum_i f^z_i \pCoplanar^{x,y}}{\sum_i f_i^z}
 =
 \left( \frac{m g^z}{m g^z - f^z_{ext}} \right)
 \left( \comP^{x,y} - \frac{c^z}{g^z} \comA^{x,y} \right)
 -
 \mathbf{S}
 \left( \frac{\tExt^{x,y} + (\comP \times \fExt)^{x,y} }{mg^z - f^z_{ext}} \right)
\end{equation}
where
\begin{equation}
 \mathbf{S} =
 \begin{bmatrix}
  0 & -1 \\
  1 & 0
 \end{bmatrix}
\end{equation}

Developing further:

\begin{equation}
 \zmp^{x,y}
 =
 \left( \frac{m g^z}{m g^z - f^z_{ext}} \right)
 \left( \comP^{x,y} - \frac{c^z}{g^z} \comA^{x,y} \right)
 -
 \mathbf{S}
 \left( \frac{\tExt^{x,y} }{mg^z - f^z_{ext}} \right)
 -
 \left( \frac{\comP^{x,y}\fExt^z - \comP^z\fExt^{x,y}}{mg^z - f^z_{ext}} \right)
\end{equation}

Grouping the terms:
\begin{equation}
 \zmp^{x,y}
 =
 \left( \comP^{x,y} - \left( \frac{c^z}{g^z} \right) \left( \frac{m g^z }{m g^z - f^z_{ext}} \right) \comA^{x,y}  \right)
 -
 \mathbf{S}
 \left( \frac{\tExt^{x,y} }{mg^z - f^z_{ext}} \right)
 +
 \left( \frac{\comP^z\fExt^{x,y}}{mg^z - f^z_{ext}} \right)
\end{equation}

\section{MPC}
The model in $x$ is:
\begin{equation}
 \begin{bmatrix}
 \dot{c}^x \\
 \ddot{c}^x \\
 \dddot{c}^x
 \end{bmatrix}
 =
 \begin{bmatrix}
  0 & 1 & 0 \\
  0 & 0 & 1 \\
  0 & 0 & 0
 \end{bmatrix}
 \begin{bmatrix}
 c^x \\
 \dot{c}^x \\
 \ddot{c}^x
 \end{bmatrix}
 +
 \begin{bmatrix}
 0 \\
 0 \\
 1
 \end{bmatrix}
 \dddot{c}^x
\end{equation}

\begin{equation}
 z^x
 =
 \begin{bmatrix}
 1 &
 0 &
 -\frac{m g^z}{(m g^z - f^z_{ext} ) \omega^2}
 \end{bmatrix}
 \begin{bmatrix}
 c^x \\
 \dot{c}^x \\
 \ddot{c}^x
 \end{bmatrix}
 +
 \frac{n^y_{ext} + c^z f^x_{ext}}{mg^z - f^z_{ext}}
\end{equation}

Rewriting:
\begin{equation}
 \begin{bmatrix}
 \dot{c}^x \\
 \ddot{c}^x \\
 \dddot{c}^x
 \end{bmatrix}
 =
 \begin{bmatrix}
  0 & 1 & 0 \\
  0 & 0 & 1 \\
  0 & 0 & 0
 \end{bmatrix}
 \begin{bmatrix}
 c^x \\
 \dot{c}^x \\
 \ddot{c}^x
 \end{bmatrix}
 +
 \begin{bmatrix}
  0 \\
  0 \\
  1
 \end{bmatrix}
 \dddot{c}^x
\end{equation}

\begin{equation}
 z^x
 =
 \begin{bmatrix}
  1 &
  0 &
  -\frac{m g^z}{(m g^z - f^z_{ext} ) \omega^2}
 \end{bmatrix}
 \begin{bmatrix}
 c^x \\
 \dot{c}^x \\
 \ddot{c}^x
 \end{bmatrix}
 +
 \begin{bmatrix}
  \frac{1}{m g^z - f^z_{ext}} &
  \frac{c^z}{m g^z - f^z_{ext}}
 \end{bmatrix}
 \begin{bmatrix}
  n^y \\
  f^x
 \end{bmatrix}
\end{equation}

The model is of the form
\begin{align}
 \mathbf{\dot{x}} &= \mathbf{A} \mathbf{x} + \mathbf{B} \mathbf{u} \\
 \mathbf{y} &= \mathbf{D} \mathbf{x} + \mathbf{G} \mathbf{f}
\end{align}

Discretization and concatenating the $x$ and $y$ DOFs leads to:
\begin{align}
 \cstate_{k+1}	=& \M{A}_k \cstate_k	  + \M{B}_k \mathbf{u}_k \\
 \cop_{k+1} 	=& \M{D}_{k+1} \cstate_{k+1} + \M{G}_{k+1} \mathbf{f}_{k+1} \\
		=& \M{D}_{k+1}\M{A}_k\cstate_{k} + \M{D}_{k+1}\M{B}_k \mathbf{u}_k + \M{G}_{k+1} \mathbf{f}_{k+1}
\end{align}
\begin{equation}
\M{A}_k =
\begin{bmatrix}
    1       & T_k   & T_k^2/2   & 0 & 0 & 0\\
    0       & 1     & T_k       & 0 & 0 & 0\\
    0       & 0     & 1         & 0 & 0 & 0\\
    0 & 0 & 0                   & 1       & T_k   & T_k^2/2   \\
    0 & 0 & 0                   & 0       & 1     & T_k       \\
    0 & 0 & 0                   & 0       & 0     & 1         \\
\end{bmatrix}
\quad
\M{B}_k =
\begin{bmatrix}
    T_k^3/6 & 0 \\
    T_k^2/2 & 0 \\
    T       & 0 \\
    0       & T_k^3/6 \\
    0       & T_k^2/2 \\
    0       & T
\end{bmatrix}
\end{equation}

\begin{equation}
\M{D}_{k} =
\begin{bmatrix}
    1 & 0 & -\frac{m g^z}{(m g^z - f^z_k ) \omega_k^2}
  & 0 & 0 & 0 \\

    0 & 0 & 0
  & 1 & 0 & -\frac{m g^z}{(m g^z - f^z_k ) \omega_k^2}  \\
\end{bmatrix}
\M{G}_{k} =
\begin{bmatrix}
  \frac{1}{m g^z - f^z_k}  & \frac{c^z_k}{m g^z - f^z_k} & 0 & 0 \\
  0 & 0 & -\frac{1}{m g^z - f^z_k}  & \frac{c^z_k}{m g^z - f^z_k}  \\
\end{bmatrix}
\end{equation}
\begin{equation}
 \omega_k = \sqrt{\frac{g}{c^z_k}}
 \quad
 \mathbf{u}_k =
 \begin{bmatrix}
  \dddot{c}^x_k \\
  \dddot{c}^y_k
 \end{bmatrix}
 \quad
 \mathbf{f}_k =
 \begin{bmatrix}
  n^y_k \\
  f^x_k \\
  n^x_k \\
  f^y_k
 \end{bmatrix}
\end{equation}

Condensing the model results in:
\begin{align}
    \begin{bmatrix}
        \cstate_1 \\
        \vdots\\
        \cstate_{N}\\
    \end{bmatrix}
    =&
    \M{U}_x \cstate_0
    +
    \M{U}_u
    \begin{bmatrix}
        \mathbf{u}_0 \\
        \vdots \\
        \mathbf{u}_{N-1} \\
    \end{bmatrix} \\
%
    \begin{bmatrix}
        \cop_1 \\
        \vdots \\
        \cop_{N} \\
    \end{bmatrix}
    =&
    \M{O}_x \cstate_0
    +
    \M{O}_u
    \begin{bmatrix}
        \mathbf{u}_0 \\
        \vdots \\
        \mathbf{u}_{N-1} \\
    \end{bmatrix}
    +
    \M{O}_f
    \begin{bmatrix}
        \mathbf{f}_1 \\
        \vdots \\
        \mathbf{f}_{N} \\
    \end{bmatrix} \\
\end{align}

or:
\begin{align}
    \cState
    =&
    \M{U}_x \cstate_0
    +
    \M{U}_u
    \dddot{\M{C}}\\
%
    \hat{\M{Z}}
    =&
    \M{O}_x \cstate_0
    +
    \M{O}_u
    \dddot{\M{C}}
    +
    \M{O}_f
    \hat{\M{F}}\\
\end{align}

expressing the ZMP to local coordinates:
\begin{align}
    \cState
    =&
    \M{U}_x \cstate_0
    +
    \M{U}_u
    \dddot{\M{C}}\\
%
    \left(
        \M{V}_0 \fp_0
        +
        \M{V} \FP
        +
        \diag{k = 1 \dots N}{\M[\hat{p}_k][]{R}}
        \CoP
    \right)
    =&
    \M{O}_x \cstate_0
    +
    \M{O}_u
    \dddot{\M{C}}
    +
    \M{O}_f
    \hat{\M{F}}\\
\end{align}

rearranging and grouping the unknowns: $\dddot{\M{C}}$ and $\FP$
\begin{align}
    \cState
    =&
    \underbrace{
    \begin{bmatrix}
        \M{U}_u    &   \M{0} \\
    \end{bmatrix}
    }_{\M{S}}
    \underbrace{
    \begin{bmatrix}
        \dddot{\M{C}}\\
        \FP\\
    \end{bmatrix}
    }_{\V{X}}
    +
    \underbrace{
    \M{U}_x \cstate_0
    }_{\V{s}}\\
%
    \CoP
    =&
    \underbrace{
    \begin{bmatrix}
        \diag{k = 1 \dots N}{\M[\hat{p}_k][]{R}^\top} \M{O}_u    &
       -\diag{k = 1 \dots N}{\M[\hat{p}_k][]{R}^\top} \M{V} \\
    \end{bmatrix}
    }_{\M{S}_{z}}
    \underbrace{
    \begin{bmatrix}
        \dddot{\M{C}}\\
        \FP\\
    \end{bmatrix}
    }_{\V{X}}
    +
    \underbrace{
    \diag{k = 1 \dots N}{\M[\hat{p}_k][]{R}^\top}
    \left(
    \M{O}_x \cstate_0
    +
    \M{O}_f \hat{\M{F}}
    -
    \M{V}_0 \fp_0
    \right)
    }_{\V{s}_{z}}\\
\end{align}

The CoM velocity is:
\begin{equation}
    \cVel =
        \diag{N}{\M{I}_{v}} \cState =
        \diag{N}{\M{I}_{v}} \left( \M{S}\V{X} + \V{s} \right)=
        \M{S}_v \V{X} + \V{s}_v
\end{equation}

\section{QP}
The Constraints are:

CoP positions
\begin{align}
  \ubarV{Z} \le \CoP \le \barV{Z} \\
  \ubarV{Z} \le \M{S}_{z} \V{X} + \V{s}_{z} \le \barV{Z}
\end{align}


Foot positions
\begin{equation}
  \ubarV{P} \le \FP \le \barV{P}
\end{equation}

The Objectives are:

Tracking a reference velocity:
\begin{equation}
    \NORME{\cVel - \cVel_{ref}} = \NORME{\M{S}_v \V{X} + \V{s}_v - \cVel_{ref}},
\end{equation}

Minimize CoM jerk:
\begin{equation}
    \NORME{\dddot{\M{C}}},
\end{equation}

Minimize the distance between the CoP positions and the centers of the feet
\begin{equation}
    \NORME{\CoP} = \NORME{\M{S}_{z} \V{X} + \V{s}_{z}}.
\end{equation}

\section{General notes on the external wrench term}


\subsection{Prediction of future values}
Knowledge of future forces and torques $f^x, f^y, n^x, n^y$ is required in the vector $\hat{\M{F}}$ and $f^z$ in the matrices $\M{D}_{k}, \M{G}_{k}$
and consequently $\M{O}_{x}, \M{O}_{u}, \M{O}_{f}$. Note, $n^z$ was implicitly ignored during the modeling.

Here, we assume knowledge of the present wrench $\mathbf{h}_0$. So a prediction model is required to obtain $\mathbf{h}_1 \ldots \mathbf{h}_N$

Some simple models can be used (mostly for testing):

Constant throughout the preview horizon
\begin{equation}
 \mathbf{h}_1 = \mathbf{h}_2 = \ldots = \mathbf{h}_N = \mathbf{h}_0
\end{equation}

Linear
\begin{equation}
 \mathbf{h}_k = m k T_k +  \mathbf{h}_0
\end{equation}

Other prediction models are also presented in the specific use cases

\subsection{Frame of reference}
Note that the modeling placed the external wrench in the CoM frame. However, $\mathbf{h}_0$ may be expressed in another frame, the hand force sensor
frames for example. For this, the wrench transformation matrix $^{com}\M{H}^{ref}$ can be used:
\begin{equation}
 \mathbf{h}_k = {}^{com}\M{H}^{ref}_k ~^{ref}\mathbf{h}
\end{equation}

where:
\begin{equation}
  \M{H}
  =
  \begin{bmatrix}
  \M{R} & \M{0}\\
  [\mathbf{t}]_\times \M{R} & \M{R}
 \end{bmatrix}
\end{equation}

\section{Case 1: Solo-carrying an object}
For an object of known mass $m^{obj}$ and location (relative to the CoM), the external wrench is:
\begin{equation}
 \mathbf{h}_k
  =
 \begin{bmatrix}
  \M{I} & \M{0}\\
  [\mathbf{t}_k]_\times & \M{I}
 \end{bmatrix}
 \begin{bmatrix}
  0\\
  0\\
  m^{obj} g^z\\
  0\\
  0\\
  0
 \end{bmatrix}
 =
 \begin{bmatrix}
  0\\
  0\\
  m^{obj} g^z\\
  y_k m^{obj} g^z\\
 -x_k m^{obj} g^z\\
  0
 \end{bmatrix}
\end{equation}
where $x$ and $y$ are the distances between the 2 CoMs (object and robot)

In general:
\begin{equation}
 \mathbf{t}_k = {}^{rob}\mathbf{t}_k^{obj} = {}^w\mathbf{t}_k^{obj} - {}^{w}\mathbf{t}_k^{com}
\end{equation}

A first approximation is to assume $x_0, y_0$ from the current configuration remains constant.
\begin{equation}
 \mathbf{t}_1 = \mathbf{t}_2 = \ldots = \mathbf{t}_N = \mathbf{t}_0
\end{equation}

A better model might be to use: the future com positions from the model $\hat{\M{C}}$
\begin{equation}
    \cPos =
        \diag{N}{\M{I}_{p}} \cState =
        \diag{N}{\M{I}_{p}} \left( \M{S}\V{X} + \V{s} \right)=
        \M{S}_p \V{X} + \V{s}_p
\end{equation}

Along with this, knowledge of the future object position in the plane ${}^wx_k, {}^wy_k$ is required. If a trajectory is available, it can be used to
obtain
this.

Without having the object trajectory, one approximation is that the reference com velocity is
perfectly imparted on the object such that:
\begin{equation}
 \hat{\V{t}} =
 \begin{bmatrix}
  x_1 \\
  y_1 \\
  \vdots \\
  x_N \\
  y_N
 \end{bmatrix}
 =
 \begin{bmatrix}
  1 & 0 \\
  0 & 1 \\
  \vdots \\
  1 & 0 \\
  0 & 1 \\
 \end{bmatrix}
 \begin{bmatrix}
  x_0 \\
  y_0
 \end{bmatrix}
 +
 \diag{k = 1 \dots N}{k T_k} \cVel_{ref}
 -
 \M{S}_p \V{X} - \V{s}_p
\end{equation}

Finally, the external wrench vector required in the QP is:
\begin{equation}
 \hat{\V{F}}
 =
 \begin{bmatrix}
  n_1^y \\
  f_1^x \\
  n_1^x \\
  f_1^y \\
  \vdots \\
  n_N^y \\
  f_N^x \\
  n_N^x \\
  f_N^y
 \end{bmatrix}
 =
 \diag{N}{\begin{bmatrix}
           -1 & 0 \\
            0 & 0 \\
            0 & 1 \\
            0 & 0
          \end{bmatrix}}
 \hat{\V{t}}
 ~
 (m^{obj} g^z)
\end{equation}

This slightly changes the ZMP formulation for the QP since $\hat{\V{F}}$ is now a function of $\V{X}$

\section{Case 2: Collaborative-carrying}
% One general thing to exploit could be giving a meaning to the weights of the QP objectives. The external wrench term only appears in the CoP and
% there is already an objective where an external CoM velocity is given. Using these 2 tasks together can be seen as a form of damping control, with the
% ratio of these task weights the damping factor.

\subsection{As leader}
For leading, a clear and independent intention is necessary. However, a good leader still needs to account for what the follower is doing, this is
where the external wrench model comes in.

% often we use the reference CoM velocity $\cVel_{ref}$.
%
% Another idea is to add the external wrench as a decision variable of the QP, trying to track the result as a target force in the whole-body control
% part

\textbf{Idea 1: Reference trajectory tracking}

If a clear trajectory is known before hand, a better objective for the leader might be a ``trajectory tracking task'' instead of just a reference
velocity . This can be formulated as:
\begin{equation}
  \NORME{
  g_m(\cAcc_{ref} - \cAcc) +
  g_b(\cVel_{ref} - \cVel) +
  g_k(\cPos_{ref} - \cPos)
  }
\end{equation}
where $g_m, g_b, g_k$ are gains. To be concise, an appropriate gain matrix can be used:
\begin{equation}
  \NORME{
  \M{G}_{mbk}(\cState_{ref} - \cState)
  }
\end{equation}

\textbf{Idea 2: Wrench in the QP decision variables}

Another idea was to place the external wrench as a decision variable of the QP. This will result in:
\begin{align}
    \cState
    =&
    \underbrace{
    \begin{bmatrix}
        \M{U}_u    &   \M{0}  &   \M{0}\\
    \end{bmatrix}
    }_{\M{S}}
    \underbrace{
    \begin{bmatrix}
        \dddot{\M{C}}\\
        \FP\\
        \hat{\M{F}}
    \end{bmatrix}
    }_{\V{X}}
    +
    \underbrace{
    \M{U}_x \cstate_0
    }_{\V{s}}\\
%
    \CoP
    =&
    \underbrace{
    \begin{bmatrix}
        \diag{k = 1 \dots N}{\M[\hat{p}_k][]{R}^\top} \M{O}_u    &
       -\diag{k = 1 \dots N}{\M[\hat{p}_k][]{R}^\top} \M{V} &
	\diag{k = 1 \dots N}{\M[\hat{p}_k][]{R}^\top} \M{O}_f
    \end{bmatrix}
    }_{\M{S}_{z}}
    \underbrace{
    \begin{bmatrix}
        \dddot{\M{C}}\\
        \FP\\
        \hat{\M{F}}
    \end{bmatrix}
    }_{\V{X}}
    +
    \underbrace{
    \diag{k = 1 \dots N}{\M[\hat{p}_k][]{R}^\top}
    \left(
    \M{O}_x \cstate_0
    -
    \M{V}_0 \fp_0
    \right)
    }_{\V{s}_{z}}\\
\end{align}

As the force is part of the ZMP expression, this might allow the robot to balance itself by applying the appropriate forces. For
safety, the applied wrench may need to be constrained
\begin{equation}
  \hat{\ubarV{F}} \le \hat{\M{F}} \le \hat{\barV{F}}
\end{equation}

and/or minimized as:
\begin{equation}
  \NORME{\hat{\M{F}}}
\end{equation}

\subsection{As follower}
For following, the robot needs to act based on the leader's intention.

% To do this, we used to change the target reference $\cVel_{ref}$ as a function
% of the sensed force (a damping-based control). With the new model, it might be possible to do the same concept but using the QP weights.

\textbf{Idea 1: Impedance control objective}

It may be possible to do planar impedance control, first defining the selection matrix $\M{S}_{xy}$ to select only the $f^x, f^y$ components. The
impedance parameter matrix $\M{G}_{mbk}$ is defined similar to the trajectory tracking task gain.
\begin{equation}
  \NORME{
  \M{G}_{mbk}\cState -\M{S}_{xy}\hat{\M{F}}
  }
\end{equation}

This could replace the reference velocity objective and the external controller to compute this reference.

\textbf{Idea 2: Force prediction models}

A better following behavior requires a ``proactive'' behavior which attempts to predict the intention of the leader. This corresponds
to predicting the future external wrench. A better prediction model should correspond to a better
proactive behavior.

Since the target application is for collaborating with humans, this model amounts to predicting the human partner's intention. One possible
assumption is to take the results presented in Bussy:IROS:2012 and infer from this 2 models to predict the forces:
\begin{enumerate}
 \item constant acceleration model - constant force
 \item constant jerk model - linear force
\end{enumerate}


\subsection{A combined approach}
Once a good leader and follower approach are formulated, it might be possible to reformulate it into a combined approach


\let\cpos\THISCOMMANDSHOULDNEVERBEDEFINED
\let\cvel\THISCOMMANDSHOULDNEVERBEDEFINED
\let\cacc\THISCOMMANDSHOULDNEVERBEDEFINED
\let\cjerk\THISCOMMANDSHOULDNEVERBEDEFINED

\let\cPos\THISCOMMANDSHOULDNEVERBEDEFINED
\let\cVel\THISCOMMANDSHOULDNEVERBEDEFINED
\let\cAcc\THISCOMMANDSHOULDNEVERBEDEFINED
\let\cJerk\THISCOMMANDSHOULDNEVERBEDEFINED

\let\cstate\THISCOMMANDSHOULDNEVERBEDEFINED
\let\cState\THISCOMMANDSHOULDNEVERBEDEFINED

\let\cop\THISCOMMANDSHOULDNEVERBEDEFINED
\let\dcop\THISCOMMANDSHOULDNEVERBEDEFINED
\let\CoP\THISCOMMANDSHOULDNEVERBEDEFINED
\let\dCoP\THISCOMMANDSHOULDNEVERBEDEFINED

\let\fph\THISCOMMANDSHOULDNEVERBEDEFINED
\let\FPh\THISCOMMANDSHOULDNEVERBEDEFINED

\let\fp\THISCOMMANDSHOULDNEVERBEDEFINED
\let\FP\THISCOMMANDSHOULDNEVERBEDEFINED

\let\fd\THISCOMMANDSHOULDNEVERBEDEFINED
\let\FD\THISCOMMANDSHOULDNEVERBEDEFINED

\let\cp\THISCOMMANDSHOULDNEVERBEDEFINED


\let\fCoplanar\THISCOMMANDSHOULDNEVERBEDEFINED
\let\fExt\THISCOMMANDSHOULDNEVERBEDEFINED
\let\tExt\THISCOMMANDSHOULDNEVERBEDEFINED
\let\gravity\THISCOMMANDSHOULDNEVERBEDEFINED
\let\comP\THISCOMMANDSHOULDNEVERBEDEFINED
\let\comA\THISCOMMANDSHOULDNEVERBEDEFINED
\let\momAngD\THISCOMMANDSHOULDNEVERBEDEFINED
\let\pCoplanar\THISCOMMANDSHOULDNEVERBEDEFINED
\let\zmp\THISCOMMANDSHOULDNEVERBEDEFINED

